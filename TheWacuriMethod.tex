\documentclass[12pt]{book}
\usepackage{graphicx}
\usepackage{verbatim}

\begin{document}

\title{The Wacuri Method: Forming Deep Connections (v0.2)}

\author{Brooks Cole, Henry Poole, Robert L. Read, and Dan Spinner}
\date{ }

\maketitle
\tableofcontents


\chapter{Forming Deep Connections}

\section{The Wacuri Method}
The Wacuri Method wakes you up to curiosity and awe.
Two or more people go on a 5-minute journey together and then discuss it.
More specifically, one person leads a journey that strives to awaken the
journeyer to awe and curiosity that can be taken into everyday life.
In so doing, it forges deep, intimate connections between beings.
A key to the method is the debrief, a discussion of the guided visualization
in which each participant has a chance to discussion the impact on their
bodily sensations, thoughts, and emotions.

\section{Not Exactly Mindfulness}

The Wacuri Method shares much in common with mindfulness and meditation.
Meditation is often lonely; Wacuri journeys are not. Wacuri journeys
seek to connect you to the community of consciousness. However,
like mindfulness, journeys seek to emancipate you from the tyranny
of everyday habit. Many of us spend each day in a long daydream, dominated
by our pasts, futures, to-do lists and our responsibilites. Wacuri seeks to wake
us up to the awesome possibilities in every moment. A journey trains you
to be present in every moment of your life. It does this by giving
you new habits of focus and
a witness outside yourself. The transformation of the journey
is made more real to you because it is affirmed by your co-journeyers.
The meditation is more intense because it is being shared with others.
Rather than rejecting society, Wacuri embraces it and asks you
to cohere to each other. Love is connection. Wacuri is
connection building. Wacuri seeks to heal the loss of connection
many of us feel.

\section{Of Cats and Trees}

A Wacuri Journey is always a journey to something which is not yourself.
It may be something inside youself, such as your inner child, or
your relationship with your father. Or it may be a journey to something
awsome like a galaxy, or a bumblebee. It seeks awe in both the awe-inspiring
and the humble. A journey seeks to give you an intimate connection
to something, like a tree or your cat, and to find awe and beauty in
that object. Because the journey is co-created, it always gets you
outside yourself. You and a friend go somewhere, and your experience
strengthens theirs, just as theirs strengthens yours.


\chapter{Wacuri Journeys}

The Wacuri Method is a structured approach to a journey. Perhaps
surprisingly, this structure gives great freedom to the journey.
Every journey evolves as a series of acts.
These are:
\begin{itemize}
\item Breathing and Posture,
\item Invitation and Invocation,
\item Introduction to Subject,
\item The Journey Proper,
\item The Apotheosis, or the Moment of Awe,
\item The Space of Appreciation, and finally
  \item The Sharing and Return.
\end{itemize}

All of these acts are accomplished in five minutes, give or take a few
minutes. The Wacuri Method compresses an ocean of awe into a
limited period of time.
Each journey includes a debrief, discussed in the next chapter,
an essential part of the Wacuri Method.

\begin{figure}
  \centering
     \includegraphics[width=0.95\textwidth]{WacuriFigures/JourneyStructure.jpg}
     \caption{Journy Structure}
  \label{fig:closeup}     
\end{figure}


\section{The Journey Jockey}

In the Wacuri Method, one of the journeyers is the Journey Jockey, a
formal role that leads the journey through each act. The Jockey
may be a formal teacher or counselor, or may simply be chosen
from the journeyers. Like anything, Jockey's may differ in
their skill, expeience, and style, but everyone has to start
somewhere. If you are reading this book, you are part of the
democracy of enlightenment, and are fully ready to jockey
your first journey.

The Journey Jockey is the only journeyer who speaks during the
first five minutes.  In the Debrief, each journeyer gets a
chance to share their sensations, emotions, and epiphanies.

The Journey Jockey speaks from the heart. Although many journeys
are educational, the Journey is not a lecture. It is not a
purely intellectual expression. The Jockey should use
rich, evocative language to try to speak to the body, mind,
heart, and even soul of the journeyers. Each journey will
play upon different aspects of the human instrument. Some
will be more intellectual, some will be more sentimental,
some more sensual. The one rule is that a Journey may not be
scripted.

Some Jockeys prefer to think of themselves as transmitting
a journey, rather than creating one. They attempt to open themselves
up to the God, or the inner light, or the quiet still voice, or
the spirit of peace, and so on. No matter how they do it,
they are co-creating the journey with the participants,
because the Jockey always imagines themselves to be in the
presence of the other journeyers, traveling with them as a guide.
The journey is literally co-created with the listeners, even
if they remain silent.

Let us now consider each Act of the Journey in turn.

\section{ Breathing and Posture}

Every journey begins with a reminder to check ones breathing and
posture. A typical opening is:
\begin{quote}
  Take a few moments to adjust your posture and still your breathing.
  Sit comfortably upright in your chair, in a posture you can
  hold for 5 minutes.
\end{quote}
As in almost every meditative practice, breathing is important, but
the Wacuri method prescribes no specific breathing. It seeks merely
to make the journey begin with an awareness of their own body--a
point of departure if you will--and prepares them to listen
intently without interruption for five minutes.

\section{ Invitation and Invocation}
The Jocky then invites the journeyers to join him or her on a journey.
The invitation is important because it is just that. The Jockey
makes no demands on the journeyers; they are free to decline going on the
journey. If they wish they can listen with a certain emotional detachment,
without allowing themselves to transported by the journery.

A typical invitation might be:
\begin{quote}
  Come with me on a journey to the Transformation of Fear.
\end{quote}
Wacuri journeys are supposed to be deep, but any given journeryer
may not be ready to go on a journey to the Transformation of Fear
of Forgiving Abusive Parents.

The Jockey may also make an invocation, a ``calling in'', of a spirit,
formally or informally. The founders of Wacuri have a sort of totem
spirit called simple The Creature that we invoke. However, an
invocation is often included in the invitation, such as:
\begin{quote}
  Come with me on a journey to the Spirit of Nelson Mandela.
\end{quote}
This formula seeks, figuratively, the guidance of the spirit of
Nelson Mandela in the journey.

\section{ Introduction to Subject}

The subject will probably already have been named, but the
Jockey now spends 15 to 30 seconds intoducing the subject. Intellectually,
this may be a few facts about the subject. Psychologically, it is
motion away from the physical surroundings of the journeyer into
space of the journey. For a brief time the journeyer is leaving
the cares of the day and the world behind. The introduction is
meant to begin this process. For many journeyers, it is a welcome
release of their own thoughts in order to give their full attention
to the Jockey for a brief time.

\section{ The Journey Proper}

The Journey proper is a timeless five minutes. That is,
the journey should transport the journeyer out of their normal sense
of time. Time is an illusion which is not needed.

However, Journeys have space. Sometimes, in the case of a
Journey to the Inner Child, this is a change in position from
one of maturity and adult responsibility through the long march
back to care-free childhood and childlike wonder.  In another case,
such as Journey to an Owl, it is a physical journey through
the chilly moonlit sky vivisected by pine brances and decorated
by the noiseless stroke of the owl's wings seeking the sound
or sight of a tasty cockroach or mouse in the mouldering duff
or the forest floor.

The Jockey should act in the spirit of transmission,
rather than in authorship of a story. In fact a Journey is
not a story, because nothing need happen. The Jockey should
be listening to their heart or soul as it recites the
story to them.

Nevertheless, the Jockey is not in a trance. Part of their
mind is thinking about how the journeyers will perceive the
journey.  They should, therefore, attempt to enrich the
experience by mentioning as many of the senses as possible.
A Journey to the Beauty of Fractals might have a little
trouble invoking the sense of smell, but in general
the more sensual the journey the better.

However, the Jockey does not need to cram too much
into a journey. Silence gives the journeyer a chance
to co-create the journey in their own minds. The oak tree
they imagine might not be quite the same as the oak tree
the Jockey imagines, but it will be more vivid for the
journeyer if they create as much of it themselves as
possible.  A good rule of thumb might be a 5 seconds
pause every 30 seconds.

\section{ The Apotheosis, or the Moment of Awe}

Although it may have several, every journey should have
at least one Moment of Awe. This is a moment when the
divine is touched, a moment of Apotheosis, or becoming divine.

Although there are many benefits to mindfulness and
collaboration, the Wacuri Method seeks above all to
awaken a sense of awe which can be taken back into the
mundane life of the journeyer to enrich it with a sense
of the awesome. A stone is a just a stone, but after a
Wacuri journey it may be a stone that generates a unique
numinour glamour.

Although there need not be a single climax to a Journey,
the sense of awe should be transmitted by the Jockey.
Hopefully the subject is something the Jockey can
truly find awe-inspiring in some way.

The Moment of Awe is emotionally and psychologically
the highest pitch of the journey. It is perhaps the
most removed from the need to do the dishes which the
journeyer will soon face in one way or another. The purpose
is not to emphasize the difference between the Death of a
Star and doing the dishes, but to allow the journey to take
some of the awesome power of a dying star back with them
to the tedious task of doing the dishes.

\section{ The Space of Appreciation}

The penultimate act of the journey of a pause that
allows for Appreciation.  Possibly there have been several
such pauses, but the most powerful journeys may build to
a climax of awe and appreciation. This requires space,
in the sense that the Jockey must pause and allow the
journeyers to appreciate the awesome nature of the subject
without the intrusion of their voice.  The journeyers
should be able to co-create the journey by imagining,
or feeling, or thinking, whatever comes from their own
hearts at this point.


\section{ The Sharing and Return}

Every Journey ends with a brief affirmation of the
shared experience and a call to gently bring the consciousness
of the listener back into the room and their own body.
The return is a coming back to Earth and in some sense
the less awesome duties of the day. Hopefully, however, the
journeyer will be in an elevated mood, or state or mind,
or spritual level.

Many journeyers find this a process that takes 30 seoconds
or more. It is often the case that the journeyers do not
with to speak for a few moments. In a sense, the gulf
between the moment of awe and the return to Earth is
so great that it cannot be passed instantaneously.

This moment is a sharing because it is a return from the
co-created journey to the fact that we are two or three
people in a coffee shop or videoconferenced together.
We may just have been three seagulls, but now we are
people with our own personalities and problems.

\chapter{The Debrief}

The Debrief is a critical part of the Wacuri Method
because it allows the journeyers to better integrate
they Journey back into their life.

After a suitable coda, the Jockey asks the journeyers
to comment on their journeys. This should begin gently
and not be rushed. Some journeyers, if there are more than
one, will not want to go first.

\begin{figure}
  \centering
     \includegraphics[width=0.95\textwidth]{WacuriFigures/DebriefStructure.jpg}
     \caption{Journy Structure}
  \label{fig:closeup}     
\end{figure}


Eventually, someone will want to speak about the journey.
The speaker is important because it is a psychological
affirmation that something has just been shared, both
for the the journeyer and the jockey.

But as the journeyers describes their experience, they
are getting something else of immense value in our world:
they are being noticed. Their thoughts matter. The group
affirms that they hear and understand their feelings
without judging them.

Of the authors of this book, some of us are very cerebreal,
some kinesthetic, and some emotional. All three ways of
experiencing a journey are valuable. It is to be expected
that not every person enjoys each journey equally, or at all.
It is furthermore the case that some people may be more easily
transported than others. It is not a contest in imagination.

The jockey, if they are comfortable with the other journeyers,
may try to elicit a emotional response from the cerebreal
journeyer, or a bodily sensation from the emotional journeyer,
and so on.

The debrief is normally betwen three and fifteen minutes. It
is possible that one person's statement will be a mere fifteen
seconds. On occation, however, the journey will be an intense
experience that excites and touchers the journeyer, and they
will want to discuss it on order to help fix it in their mind.

The act of discussing the journey, just like the act of keeping
a dream journal, makes the journeys more memorable by allowing
the verbal part of the brain to register and record the often
intense feelings and elations of the journey.

\begin{figure}
  \centering
     \includegraphics[width=0.95\textwidth]{WacuriFigures/JourneyerInteraction.jpg}
     \caption{Journy Structure}
  \label{fig:closeup}     
\end{figure}


In listening to people's reactions to journeys, each participant
gets an intimate glance into the personality, thoughts and feelings
of the other journeyers. Even though there is no need for a journeyer
to comment on another's journey, a shared sense of camaraderie is forged.
It is often suprising to observe how different another's reaction
can be. When reactions are similar, it creates a sense of like-mindedness.

After participating in several journey's with the same person, you
start to feel that you know and understand the other person, and
likewise that they have been shown a part of you which you would
not ordinarily share with strangers. The fundamental human need
to be understood and known is satisfied by this connection.

The debrief also gives feedback to journey jockey. Although that is
not its purpose, it allows a the jockey to become more skillful
at producing impactful and uplifting journeys. This aspect of
co-creation further democratizes the entire experience, because
the jockey is not seen as an authority figure but rather a
perfomer participating in an closely shared experience.

The final act of the debrief is a celebration of this shared experience.
This recognition that an act has been shared and that each person
has spoken and been heard is an affirmation of the impact of the
journey which allows the jouneyer to carry this short celebration
into the rest of their day, hopefully remembering a sense of awe
and connected to other human beings.


\chapter{A Sample Journey}

\section{The Consciousness of Cells}

{\em


Take a couple of deep breaths in your own rhythm.

Adjust your posture to be comfortable.

Come with em today on a journey to the consciousness of cells.
I want you to imagine
cells throughout your body. A few, a lot. One location, several location.
These extraordinarily tiny, tiny creatures that hold everyting that together, that
are one of the critical units of our entire structure.

Imagine your cellular structure, somewhere in your body, everywhere in your body.
Millions upon millions of cells. Interacting, sharing information, nutrients,
exchange.

And now I want you to imagine that somehow a few of them---just a few---start to light up.
In other words, they are aware that you observe them.
They lighten up, the light of consciousness.
See them perhaps in clusters, in one part of your body or another.
Or perhaps many cells, but at least a few. And take a momemnt now
to see your cells lighten up in recognition of you and your growth in
your life.

PAUSE

Perhaps now, the number of cells and the location of cells that are lighting
up, that are becoming conscious, or perhaps that you are aware are conscious
is increasing. Somehow consciouness begets consciouness. See if you cells
now are multiplying their light, feeding one another so to speak, resonating
with every light vibration one to another. Feel your body becoming more
alive, more alert, as your cells wake up one by one.

And now these clusters of lit up cells are becoming more and more, almost
as if there was a rhythm building.
in a wave pattern. Feel many more cells lighting up. Somehow connected,
communicating with one another.

PAUSE

And now, they are all lighting up. Every single cell in your physcial
body---lighting up, celebrating you, celebrating their own awareness.
Joining you in your growing awareness, bringing health, clarity,
strength and beauty. Just sit soaking that up for a moment.

PAUSE

Feel the vibrancy of it all. Reaching a peak. Forever changed.
Bright, bright as you can imagine.

SHORT PAUSE

And when you are ready, give thanks, come back into the room,
and have a wonderful day.

}

\hrulefill



You have just read a transcript of an actual journey entitled the
Consciousness of Cells, jockeyed by Dan Spinner in 2014. The
whole journey recording is 5 minutes 17 seconds long. It was
performed without a script. Like all such human speech, it
is somewhat broken. It is cogent, but does not always
use complete sentences. Although it lacks the power of Dan's
voice, we hope the transcript gives you an idea of a journey,
although the content varies quite widely. We hope this
encourages you to try your own.

Note also, that although perhaps educational to someone
who has not heard of the cellular theory of life, it is not
a lecture on biology, but rather a visualization that Dan
transmits from his own way of thinking directly to the journeyers.

To demonstrate the seven acts of a typical journey, we
now repeat the journey, intermixing comments.


\hrulefill

\begin{quote}{\em
Take a couple of deep breaths in your own rhythm.

Adjust your posture to be comfortable.
}
\end{quote}

Note here that Dan directs our attention to our breathing
and posture to both prepare for the five minute journey
and to bring our consciouness into the body

\begin{quote}{\em
  Come with me today on a journey to the consciousness of cells.
}\end{quote}

The Invitaiton prepare the journeyer mentally and mentions
the subject. The journey may choose to decline the invitation.
In this journey, there is no invocation.
\begin{quote}{\em
I want you to imagine
cells throughout your body. A few, a lot. One location, several location.
These extraordinarily tiny, tiny creatures that
hold everyting that together, that
are one of the critical units of our entire structure.
}\end{quote}

Here Dan has introduced the subject, which sets the stage and
begins transporting the journeyer out of everyday consciousness
and into an imagined space. Not that so far the cells are
static---they are not doing anything, they are just there.

\begin{quote}{\em
Imagine your cellular structure, somewhere in your body, everywhere in your body.
Millions upon millions of cells. Interacting, sharing information, nutrients,
exchange.
}\end{quote}
Now the journey is beginning as the cells start to act. The journeyer
must use their own imagination to try to picture this.

\begin{quote}{\em
And now I want you to imagine that somehow a few of them---just a few---start to light up.
In other words, they are aware that you observe them.
They lighten up, the light of consciousness.
See them perhaps in clusters, in one part of your body or another.
Or perhaps many cells, but at least a few. And take a momemnt now
to see your cells lighten up in recognition of you and your growth in
your life.
}\end{quote}

The journey now is fully underway. Hopefully the journeyer is completely
transported out of the mundane thoughts of their everyday tasks. This
journey is richly visual, allowing the journeyer to exercise their
own imagination.

\begin{quote}{\em
  PAUSE
}\end{quote}

To give time to mentally construct this image, the jockey pauses.
After a respectful time, Dan begins again:

\begin{quote}{\em
Perhaps now, the number of cells and the location of cells that are lighting
up, that are becoming conscious, or perhaps that you are aware are conscious
is increasing. Somehow consciouness begets consciouness. See if you cells
now are multiplying their light, feeding one another so to speak, resonating
with every light vibration one to another. Feel your body becoming more
alive, more alert, as your cells wake up one by one.
}\end{quote}

Dan has now brought in a sense of motion and vibration. A somatic
component is added with the suggestion to ``Feel your body becoming...''.

\begin{quote}{\em
And now these clusters of lit up cells are becoming more and more, almost
as if there was a rhythm building.
in a wave pattern. Feel many more cells lighting up. Somehow connected,
communicating with one another.
}\end{quote}

Although Dan mentions no sound, he invites the journeyer to
imagine a rhythmic wave pattern, further enhancing the journey.
However, Dan never specifies what color the light given off is.
To one, it might be white, to another golden, to another different
hues depending on where they are in the body. The jockey is not
attempting to completely describe the experience, but to
transit ideas and feelings. Here Dan once again pauses before
continuing.

\begin{quote}{\em
  PAUSE



And now, they are all lighting up. Every single cell in your physcial
body---lighting up, celebrating you, celebrating their own awareness.
Joining you in your growing awareness, bringing health, clarity,
strength and beauty. Just sit soaking that up for a moment.
  }\end{quote}

Dan is building to a Moment of Awe. He is brought in an emotion,
that of celebration, and the intellectual idea of awareness.
Positive imaginations of health and clarity are invited,
and then he pauses again.

\begin{quote}{\em
PAUSE

Feel the vibrancy of it all. Reaching a peak. Forever changed.
Bright, bright as you can imagine.
  }\end{quote}

Dan has now reached the Moment of Awe. He is asking the journeyer
to imagine as intensely as possible, and slyly suggesting that this
change, which is only been imagined, will outlast the the journey
as he says ``Forever changed.''

\begin{quote}{\em
  SHORT PAUSE
  }\end{quote}

Dan allows the final Space for Appreciation. He gives time
for the journeyer to imagine a potentially lasting
visual, emotional and intellectual impression.

\begin{quote}{\em
And when you are ready, give thanks, come back into the room,
and have a wonderful day.
}\end{quote}

This is the Sharing and the Return. Dan gives the journeyer
permission to take some time, but reminds them to give thanks.
He explicitly guides them down from the peak experience back
``into the room'', which also means ``into your normal, but
perhaps elevated concsciouness. Finally, the journeyer is
asked to ``have a wonderful day'', a formula which ends
the timeless nature of the journey, in which hopefully
normal time has stopped, and restarts the journeyers normal
perception of time.

\hrulefill

Note that this journey might not be perfect---in fact Dan himself
scored it an 8 on a scale of 1 to 10. The quality of the journey
may or may not be imperfect, but it is better that the
journey be genunie and spontaneously transmitted from the
hear than scripted. If the journeyers note imperfections
in the jockey, it enhances the expeience, just as a live
music performance is more engrossing than a studio performance,
although the studio performance is in a sense more carefully crafted.

\chapter{Sample Debriefs}


Journey to the Heart Center

Dan: ``You want me to do it agiain?''

Adam: ...you know I don't thank that is necessary, I think I got the transmission.
I noticed that my heart center it didn't have a, uh, it had a
certain density in the middle, that went further in, it got more dense,
there was no real delineation like it moved out past my shoulders
out in front of me, beside me.
Kind of whitish in color on the outside and yellowish where it became more
dense. Certainly interacted at the same beat that physcial heart was
beating. Certainly gave me a feeling of warmth and a certain feeling of being
connected to other people, not all things, but other people.

``What woud you say your emotional state was or is?''

Very calm. When you said think of someone you love I noticed
that swirls of red and blue, like sort of Pollock-swirls,
went inside of it and like, um, a combination of joy and
sadness without thinking of anyone in particular.

Dan: ``Yeah. Henry?''

Henry: When I went there, I thought, it kind of started out as white,
and it quickly turned green, kind of green glowing, like kind
of sphere, but kind of a star though with points coming out at
90 degrees all the way around,  coming out on the top and bottom, kind of like a star.
It kept changing colors too to yellow and blue and becoming larger, and um,
and it was pulsing similar I think to my heart, as well. It got larger,
larger than my body, larger than the planet, I felt like it was out
beyond the universe, heh, it just seemed like was everything.

When I thought about Maria's energy and her heart, I got to think about someobody...
love her, I just,  I felt it, um, I felt calm, I still feel calm. I feel like I'm floating.
In joy...it's kind of a joy feeling of just being connected. Thinking about bringing it
into my day, I'm like, just awe, yeah, I like that, I like that, I want to see that, I want feel that.

``I should have known Henry when I found myself saying make it has large as you want,
you would make it as large as the universe, heh heh.''

Dan: And you know, when we practice these centers, this one and other, another act of integration,
taking the energetic aspects of our beings, exploring them and integrating with our psyches and our physcial bodies,
and so the act of taking into the workplace or with a loved one, either in or near our reality or imagination changes things.
For example, just try to imagine if you can, being in your heart center and being mad at someone. Or annoyed. Um, it won't happen.

Well the other way of being annoyed or mad at someone but opening up your heart center for the annoyance or anger.
When that other person colleague, friend, partner, learns about the heart center, then good for them, then think about,
just as you implied Henry, think about the power doing that with your family of your kids. Imagine teaching your
kids about the heart center. There are many, many applications, it's just fun to explore.

Dan: Other comments or questions for one another?

LONG PAUSE

Adam: Nothing is coming up.

Dan: Think about your own relationship. And the homework is to try it. Just play with it. Maybe when you are in a pretty good place,
but when you are not in a good place, you might want to try it to.

Dan: How are you each feeling now?

Henry: I'm feeling you know, just sedate.

Adam: Pleasant and a little bit excited to try this out both with my daughter and with a couple of friends of ours.

Henry: Maybe I'll try it with one or both of my boys.




\chapter{How to Use Journeys}


Create a way of measuring and talking about deep connections. Wacuri
is an algorithm for deepening connections. Want a new glossary and
words for the formation of connections.

Image of person as black hole becoming a field of stars. Create good
stories for this

Create a ability to measure growth through healing or for specific
purpose.

Analyze language of debrief.

\chapter{ A Multiverse of Journeys}

Many traditional mindfulness practices recommend performing the
same exercise every day. The Wacuri Method supports the
spontaneous creation of new subjects each time, within the
basic structure of the method.

Although this lack of discipline may at first seem a weakness,
we have found it to be a strength. Unlike a mindfulness practice
that seeks ``one-pointedness'' or to still the toughts completely,
the Wacuri Method enourages rich exercise of the mind. Like
traditional mindfulness, the journeyers experience is non-verbal
until the debrief. The jockey, of course, is constructing a
verbal experience.

If this is indeed a strength, we suggest you explore it
fully. There are no limits to the subjects of the meditations.
We often use objects or animals, such as a Bumblebee or a
Spiderweb from nature because the
tend to invoke awe. One of of the authors (Rob) is a
computer scientist who sometimes does journeys to abstract,
non-physical subjects such as the Realm of Mathematics.
Some of our most powerful journeys are psychological, such
as Journey to the Inner Child or Journey to the Transformation
of Fear.

By celebrating the diverstity of such subjects, it is necessarily
the case that not every journey will resonate with every journeyer.
In general each of us takes varying of pleasure and exhiliration
along different dimensions of our psyche from each journey.
Not every expeience will be a peak experience. Sometimes in
the debrief we express that the journey was mildly interesting
only to disover that the same journey riveted another journeyer,
perhaps due to their past experience or a difference in their
personality.

Appendix \ref{sec:journeys} lists some of the journeys which we have actually
produced. However, feel free to use these topics yourself. Your
understanding of Dark Matter or the Inner Child may be completely
different than ours. Because journeys are not lectures meant
to convey information, it is not particularly important that
a journey cover or not cover a particular topic. The value
of the journey lies solely in its experience and effect.

\chapter{Origin of Waking Up Curious}

In early May, 2013, Dan Spinner was in Vancouver Canada and became
curious about using technology to improve and broaden his coaching
practice by using modern technology. While meditating, he visualized
collaborating with an expert.

At the same time, several hundred miles south in Lafayette CA, Henry
Poole was becoming curious about bringing mindfulness practices into
government. Henry had just met with executives at the FCC and Census
and felt that the modernization of government technology was hindered
by culture. Adopting new technologies was crippled by the
intransigent, hardened, blame-oriented institutions within the Federal
Government.

On May 9th, 2013, Henry emailed Dan to discuss possibilities to
transform the culture of the US Federal Government.

They quickly arranged a phone conversation, where Henry told Dan that
he was trying to bring enlightenment to, of all places, the US
government. Dan shared his interest in increasing the impact of his
work through emerging technology. At that moment, Henry and Dan began
weekly calls to brainstorm ideas. They discussed the need for a new
language, new tools, and new science. They became very excited about
the possibilities.

Looking for ideas to transform government, Henry and his board at
CivicActions had also recently enrolled in a course at the Google
spinoff - Search Inside Yourself Institute (SIYLI). SILYLI had
developed a very effective methodology for increasing the productivity
of programmers by teaching meditation practices. One of Henry's key
takeaways from the SILYLI training was the interest in Curiosity.

During the weekly calls with Dan, Henry became clear that the struggle
that federal government employees were experiencing, while much more
aggrivated, was similar to his own. He felt too busy to keep up his
daily meditation practice. For Henry, the logic and obvious benefits
weren't enough to get out of his personal daily busy habits. He knew
the benefit of a daily practice would pay for itself immediately...but
just couldn't break his habit of back to back meetings, endless todo's
and dealing with two boys entering their teenage years.

He asked Dan to coach him, with a difficult limitation: he wanted to do five minute meditations. Furthermore, Henry needed a coach
or a mindfulness buddy, much as people need running or lifting partners, to make them more likely to do their training by adding the peer
pressure and social facilitation of doing something collaboratively. Few people will let a partner down by not showing up non-challantly.

Dan was a life coach who had meditated for years in the
20 minute or more style.  In fact, there is an unstated
belief in the mindfulness commmunity that ``more hours makes
you a more better person.'' Five minutes was quite a
departure for his traditional practice.

In one of the weekly meetings, Dan and Henry discussed this 5 minute requirement. Henry knew that he just wouldn't commit to a longer block of time. Dan wasn't sure that he could do it but agreed to give it a try. 

Dan rose to this challenge by employing one of his firmly held convictions: that time is an illusion.
Perhaps the twenty minute rule-of-thumb was a guideline that could be questioned.
By not planning or scripting the meditation but rather spontaneously transmitting
the visualization, Dan and Henry found that they could
make an effective journey in only five minutes.
Dan had always asked the groups he coached to comment on
their meditation experience, but now, because Henry needed a
meditation partner, they realized they could make the debrief
an essential part of every mindfulness training session.
Although begun as a crutch for a busy executive, it turned
out that having a person there to share the experience deepened
the experience by forcing both a human connection and a verbalization
of the experience.

For years, both Henry and Dan had been practicing meditation. They
both noticed that a regular practice of meditating brought almost
magical connections into their life. They both noticed that
maintaining a calm state of centeredness brought more frequent high
quality insights. There was a clarity that emerged, where their
decisions seemed more accurate. They became more curious.

They had also both experienced lucid dreaming. A lucid dream is
defined in wikipedia as "a dream during which the dreamer is aware of
dreaming. During lucid dreaming, the dreamer may be able to exert some
degree of control over the dream characters, narrative, and
environment. Henry had a desire to bring aspects of lucid dreaming
into his waking state.  What he wanted now was to wake up from his
tedious overfilled day. Just as one can wake into a dream and realize
one has the power to fly, or change the grizzly bear into a teddy
bear, or to do anythign in the dream state you want, so to can a
person wake up in their daily life and realize that they are not
imprisoned by their todo lists and meeting schedule. Responsibility
does not preclue freedom.

One of the techniques used by lucid dreamers is to create a personal
anchor that once observed in a dream will trigger the dreamer to get
that they are observing the dream. That anchor puts the dreamer is a
state where they are aware of their power. Henry thought that perhaps
the waking state could be similarly hacked. Maybe the busy life of
never ending thoughts could be interupted by a noticing that he was
not his thoughts. Maybe practicing quick meditations could create some
similar anchor where he could wake up to curiousity in daily living.

Perhaps, one can wake up to vast potentialities in normal life. The name Wacuri is a portmanteau of ``Waking Up Curious.''

Dan, a student of traditional enlightenment practices, knew this
as stepping into the unknown, and used it in several ways
in the Wacuri Method.  For example, in the semiweekly meeting of the founders of
Wacuri, the person who will be the jockey and the topic of the meditation
is chosen spontaneously, on the
spot. Sometimes the jockey does not choose the topic. By nonjudgementally
allowing a spontaneous journey, it is possible to have an
effective, if sometimes fumbling, collaborative meditation.

Through subsequent years of practice, the Wacuri Method was developed
into a set of best practices and guidelines presented here. The system
fundamentally was born from the necessity of a personal growth system
applicable to busy executives.
\begin{comment}

The Wacuri Method began when Dan Spinner became curious
if he could get use technology to improve and broaden his
coaching practice by using modern technology. He places a
phone call to experts and told them he wanted to understand
this internet thing.

Three weeks later Henry Poole called him. Henry was
trying to bringing enlightenment to, of all places, the
government. Henry had been at SILLY, the Search Inside Yourself
Institute, and had been inspired by the mindfulness exercises and themes
they taought. He sought a way to bring these meditative practices into
the intransigent, hardened, blame-oriented institutions within
the Federal Government. Henry knew that to do this he would
have to work on himself first. But Henry was a software executive
with two growing boys and an impossibly busy life. He asked
Dan to coach, with a difficult limitation: he wanted to do
five minute meditations. Furthermore, Henry needed a coach
or a mindfulness buddy, much as people need running or lifting partners,
to make them more likely to do their training by adding the peer
pressure and social facilitation of doing something collaboratively.
Few people will let a partner down by not showing up non-challantly.

Dan was a life coach who had meditated for years in the
20 minute or more style.  In fact, there is an unstated
belief in the mindfulness commmunity that ``more hours makes
you a more better person.'' Five minutes was quite a
departure for his traditional practice.

Dan rose to this challenge by employing one of his
firmly held convctions: that time is an illusion.
The twenty minute rule-of-thumb was a guideline that
could be questioned.
By not planning or scripting the meditation but rather spontaneously tranmitting
the visualization, Dan and Henry found that they could
make an effective journey in only five minutes.
Dan had always asked the groups he coached to comment on
their meditation experience, but now, because Henry needed a
meditation partner, they realized they could make the debrief
an essential part of every mindfulness training session.
Although begun as a crutch for a busy executive, it turned
out that having a person there to share the experience deepened
the experience by forcing both a human connection and a verbalization
of the experience.

Henry's personal desire, which he found resonated with
the employees of his own firm, was to be more curious.
Henry wanted to wake up to curiosity, of which the peak
experience of awe is formalized as a critical part of the Wacuri Journey.
In studying lucid dreaming, Henry had learned, through practice, to wake up while
in a dream state. What he wanted now was to wake up from his
tedious overfilled day. Just as one can wake into a dream and realize
one has the power to fly, or change the grizzly bear into a teddy bear,
or to do anythign in the dream state you want, so to can a person
wake up in their daily life and realize that they are not imprisoned
by their todo lists and meeting schedule. Responsibility does not
preclue freedom. One can wake up to vast potentialities in
normal life. The name Wacuri is a portmanteau of ``Waking Up Curious.''

Dan, a student of traditional enlightenment practices, knew this
as stepping into the unknown, and used it in several ways
in the Wacuri Method.  For example, in the semiweekly meeting of the founders of
Wacuri, the person who will be the jockey and the topic of the meditation
is chosen spontaneously, on the
spot. Sometimes the jockey does not choose the topic. By nonjudgementally
allowing a spontaneous journey, it is possible to have an
effective, if sometimes fumbling, collaborative meditation.

Through subsequent years of practice, the Wacuri Method was developed
into a set of best practices and guidelines presented here. The system
fundamentally was born from the necessity of a personal growth system
applicable to busy executives.

\end{comment}

\chapter{Stories}

\section{On Mortality}

Dan's story of addressing 40 cancer patients.

\section{Imaginal Calisthenics}

Brook's personal experience strengthening imagination through practice.

\section{Henry's Experience}

Placeholder 

\section{A Field of Stars}

A fictional story about someone moving from lonely to connectedness via journeys.
\chapter{The Wacuri Method and Technology}

\chapter{Transmission}

We Wacuri Method requires at least two journeyers. One person creates the
spoken Journey and it called the Jockey.

To our way of thinking the Journey is not written but transmitted.
A Journey is a performance, not a composition.
The Jockey attempts to let the topic of the meditation become the
source of the feelings, thoughts, and impressions which make up the journey.
This chapter is devoted to explaining this process and giving some of the
best practices and techqniques we have learned to help you Jockey your own
Jouneys.

The goal of being a journey is to awaken curiosity and if possible awe
in the journeyer. You could do this by writing, or by drawing, or
with a photograph. However, the Wacuri Method uses a different
technique. Rather than using the practices of creative writing,
or slam poetry, or oral story telling, all of which are beautiful
and powerful arts, the Wacuri Method sees the Journey as flowing
form the source through the Jockey to the journeyer in a process
called {\em transmission}.

In this process, rather than intellectually constructing a spoken-word
experience, the goal is to authentically convey with minimum artifice
the thoughts, feelings, and awe of the source directly. It is
perhaps closer to a jazz improvization than playing from a
written piece of music.

To do this, the jockey should center themselves and attempt
to obtain a so-called flow state. It may help to invoke something
larger than yourself. You then open yourself to the
topic of the meditation. At that point the topic becomes
the source---the active subject of the meditation, and
the source of the conveyed or transmitted thoughts, adumbrations,
impressions, sensations and feelings.

If the
source is a tree, the jockey should try to feel the
tree intensely. A pracice that sometimes help is to recall
the most intense memory you have of a tree---perhaps a
favorite tree that you climbed as a child. You
should try to cohere to the tree. If the object
is something of which you can have no direct
experience, such as a black hole, you should still
try to connect to the imagined power, energy, majesty
and danger of a black hole. It is better to
do this without verbally listing too many
aspsects of the tree or black hole in your mind.
You are not about to compose a lecture, but to
transmit impressions from the source.

The may be emotionally challenging as well as
mentally difficult.  For example, if the
source is the Transformation of Fear or
Recovery from Addiction, the jockey must feel
the fear or chains of compulsion, and be prepared to
convey that, hopefully before conveying relief
from the fear or emancipation from compulsion.

After mentally connnecting to the source,
the jockey must connect to the journeyers.
Even if making a recording, the jockey must
imagine the journeyers and begin to see
themsleves as a conduit or vessel for transmission
from the source to the journeyer.

Trying to feel both the source and journeyer,
the jockey is ready to begin.

We have already outlined the basic structure
of the journey, which should be considered
an important guideline.
However, other tips to keep in mind include:
\begin{itemize}
\item Invite you journeyers to vividly
  connect to the source. It may help to use language
  that mentions the senses and emotions or the
  psychological structure of your journeyers.
\item Frequently give the journeyers permission
  to construct their own version of the source
  by saying ``You choose...'' or ``You pick...''.
  Precision quickens writing, but is not needed in
  transmission, though you may find yourself giving
  precise details, while leaving some aspects of the
  source unspoken or unspecified.
\item Try to transmit partially verbally
  and partially emotionally. It is not necessary
  to reduce all information to words.  Beginners
  generally find the source gives them an avalanche
  of words that far exceed what can be transmitted.
\item Silences are golden and necessarily. Pauses
  are needed for your journeyers to have time to
  connect to the source in their own imagination.
  Pauses also give you a chance to select the strongest
  impression to transmit from the source.
\item Fill silences with emotion, not sound. When
  you pause, you should still feel your connection to the
  source as compellingly as possible and imagine this
  same connection to your journeyers.
\end{itemize}

Just as you should love someone not just when you
say ``I love you!'', but before and after this
exclamation you should try to connect to your
source and journeyers ahead of time and follow
through with some mental energy after the journey.
This does not have to be specific. For example,
you may not know the topic ahead of time, but
you can still imagine a successful connection
between the source and journeyers.


Finally, you may want to watch for
delightful surprise as a marker for your
success. If in a journey to a Flower you
find yourself delightfully surprised by
something you have said that appears unplanned,
perhaps the life of a spider residing in the
Flower, this may be a sign that you have
acheived the spontaneity that you are seeking.





\chapter{Curiosity}

Wacuri seeks to awaken people to presence in their daily
lives so they are curious about things they may once have taken for
granted. Journeys may rmind people of and restore them to the child-like
wonder they once know. Wonder and curiosity encourage deeper connections
between people and the objects of their curiosity.

Ideally you should be curious about the journey, your journey partners,
and your self. Curiosity dissolves the ego. Overtime, journeys strengthen the ``curiosity muscle'',
which can be found working in opposition to the self-centeredness muscle.
Mindfulness subtracts distraction, Wacuri adds curiosity, which is contagious and addictive.

In order to be curious, people must feel safe. The jockey and the journeyers must
support in each in creating an emotionally safe environment. For this reason, it
may be that the best size for a journey taken with strangers is only two or three people.
Everyone seeks connections to other people, when they are able to manage the risk associated with
forming those connections. The journeyers should all help each other to feel and be safe.

\chapter{A Journey Journal}

Experience cannot be reduced to a number.  Nonetheless, just
as many athletes keep a training log, some people will find
keep a Journey Journal a pleasant and informative experience.
We recommend simplicity. Every entry in the Journey Journal
needs only five items:
\begin{enumerate}
\item the date,
\item the journey title,
\item a selection of words from a mood circumflex that
  describe your mood,
\item a number between 1 and 10 representing the quality of
  the journey experience, and
\item a free-form field where you can write any comments
  you want about the journey.
\end{enumerate}

You may choose to select one or more words from a standard
emotion circumplex like that shown in Figure \ref{fig:emotionwheel}.
\begin{figure}
  \centering
     \includegraphics[width=0.95\textwidth]{WacuriFigures/EmotionWheel.jpg}
     \caption{Emotion Wheel (Copyright not yet obtained for this draft.)}
  \label{fig:emotionwheel}     
\end{figure}


As a convenience, we have provied one page of such a Journey Journal here.

\begin{tabular}{ |p{2cm}||p{2cm}|p{3cm}|p{1cm}|p{5cm}|  }
 \hline
 \multicolumn{5}{|c|}{Journey Journal} \\
 \hline
 Title& Date & Mood Words & Q (1-10) & Comments\\
  \hline
 \hline
 & & & & \\
 \hline 
 & & & & \\
 \hline 
 & & & & \\
 \hline 
 & & & & \\
 \hline 
 & & & & \\
 \hline 
 & & & & \\
 \hline 
 & & & & \\
 \hline 
  & & & & \\
 \hline
  \hline
\end{tabular}



\chapter{Related Systems}

The Wacuri Method overlaps with mindfulness.  Mindfulness is focused
on taking things away. For example, it stives to rein in the
``monkey mind'' or the ``yapper'' that constantly intrudes with
verbal thoughts. The Wacuri Method, on the other hand, seeks
to awaken curiosity. Both systems strenghten the attentiveness
and powers of concentration, as represented in Figure \ref{fig:wacurivsmindfulness}. Mindfulness subtracts, Wacuri adds.

\begin{figure}
  \centering
     \includegraphics[width=0.6\textwidth]{WacuriFigures/WacuriMindfulnessDiagram.png}
     \caption{Wacuri vs. Mindfulness}
  \label{fig:wacurivsmindfulness}     
\end{figure}

These are loose ideas and hypotheses that we have about metrics and connectedness:
\begin{itemize}
\item Five minutes more frequently is as valuable (or more) than longer meditations.
\item With practice, the time to get into a zone of beneficial mental state decreases.
\item Practice with the Wacuri Method increases curiosity, and, necessarily, presence.
  \item The impact of transmissions is scalar and may vary.
  \end{itemize}

\appendix

\chapter{Some Journeys}
\label{sec:journeys}

Some of our best journeys
\begin{enumerate}
\item A Galaxy Cluster.mp3
\item Forgiveness (1).mp3
\item Journey to a Bumblebee (2).mp3
\item Journey to A Spiderweb+Music.mp3
\item Journey to Ancient Bacteria .mp3
\item Journey to Bioluminescence.mp3
\item Journey to Dark Matter and Dark Energy.mp3
\item Journey to Dark Matter and Dark Energy+Music.mp3
\item Journey to Forgiveness.mp3
\item Journey to The Birth of Stars+Music.mp3
\item Journey to the Birth of Stars.mp3
\item Journey to the Consciousness of Cells-MP3 File.mp3
\item Journey to The Elders+Music.mp3
\item Journey to the Illusion of Time .mp3
\item Journey to The Magnetic Field of the Earth+Music.mp3
\item Journey to The Magnetic Field of the Earth.mp3
\item Journey to The Song of Owls+Music.mp3
\item Journey to the Transformation of Fear.mp3
\item Journey-to-the-Elders.mp3
\end{enumerate}



\begin{comment}
UNEDITED
\begin{enumerate}
\item Journey to a Golden Atom
\item Journey to a Fellow Creature
\item Journey to the Grid Work
\item Journey to the Heart Centre
\item Journey to The Inner Child 
\item Journey to Nature
\item Journey to the Spirit Mandela 
\item Journey to My Future Self 
\item Journey to the Other
\item Jounrney to the Sea of Possibilities
\item Journey to Space
\item Journey to the Body
\item Journey to the Void
\item 2014february10
\item 2014febuary20
\item 2014march04
\item A Galaxy Cluster
\item atom
\item fellow\_creature\_raw.aup
\item Forgiveness (1)
\item Forgiveness
\item gridwork\_raw.aup
\item heart\_center\_raw.aup
\item inner\_child\_raw.aup
\item Journey to a Bumblebee (2)
\item Journey to a Fallen Leaf
\item Journey to a New Leaf
\item Journey to a Photon(1)
\item Journey to a Rose
\item Journey to a Spider Web
\item Journey to an Asteriod
\item Journey to Ancient Bacteria 
\item Journey to Bioluminessence
\item Journey to Dark Matter and Dark Energy
\item Journey to Nature Spirits
\item journey to nature
\item Journey to Photosynthesis
\item Journey to Silence and Breath
\item Journey to the Aura of all Things 
\item Journey to the Birth of Stars
\item Journey to the Center of Your Own Being
\item Journey to the Consciousness of all Things
\item Journey to the Emergence of a Flower
\item Journey to the Eternal Now
\item Journey to the Idea of Death
\item Journey to the Illusion of Time 
\item Journey to the Kingdom of Mosses and Lichen
\item Journey to the Mineral Kingdom
\item Journey to the Northern Lights
\item Journey to the Song of Owls
\item Journey to the Sound of Music Everywhere
\item Journey to the Sound of Rain
\item Journey to the Spirit of Beauty in all things 
\item Journey to the Spirit of Beauty
\item Journey to the Spirit of Buddha
\item Journey to the Spirit of Celebration
\item Journey to the Spirit of Robin Williams
\item Journey to the Thousand Petalled Lotus Flower
\item Journey to the Transformation from Hectic to Harmony
\item Journey to the Transformation of Confusion to Clarity
\item Journey to the Transformation of Fear
\item Journey to the Transformation of Sadness and Ennui
\item Journey to Your Monkey Mind 
\item Journey to Your Neural Network
\item Journey to Your Skeletal Structure
\item Journey-to-the-Elders
\item mandela raw.aup
\item my\_future\_self\_raw.aup
\item my\_future\_self\_raw\_data
\item e00
\item other
\item rec\_+13233934046\_07\_Apr\_2014\_12\_10\_55
\item rec\_+13233934046\_08\_Apr\_2014\_09\_28\_36
\item rec\_+19492021057\_03\_Apr\_2014\_08\_57\_00
\item rec\_+19492021057\_10\_Apr\_2014\_09\_19\_58
\item sea\_of\_possibilities\_raw.aup
\item 
\item A Galaxy Cluster
\item Forgiveness (1)
\item Journey to a Bumblebee (2)
\item Journey to Ancient Bacteria 
\item Journey to Bioluminessence
\item Journey to the Birth of Stars
\item Journey to the Illusion of Time 
\item Journey to the Transformation of Fear
\item Journey-to-the-Elders
\item space
\item Spirit of A Mother
\item The Fractal Universe
\item The Magentic Field of the Earth
\item the\_body
\item the\_void\_raw.aup
\item to a Future Self
\item Wacuri Journey 082014 - Mother
\item 
\item 
\item 
\item NOTE: At 11/11/2015, this section seems to be up to date.
\item fellow\_creature\_raw (1).aup
\item fellow\_creature\_raw.aup
\item gridwork\_raw (1).aup
\item gridwork\_raw.aup
\item heart\_center\_raw (1).aup
\item heart\_center\_raw.aup
\item inner\_child\_raw (1).aup
\item inner\_child\_raw.aup
\item mandela raw (1).aup
\item mandela raw.aup
\item my\_future\_self\_raw (1).aup
\item my\_future\_self\_raw.aup
\item my\_future\_self\_raw\_data
\item sea\_of\_possibilities\_raw (1).aup
\item sea\_of\_possibilities\_raw.aup
\item the\_void\_raw (1).aup
\item the\_void\_raw.aup
\item rec\_+13233934046\_07\_Apr\_2014\_12\_10\_55.mp3
\item rec\_+19492021057\_03\_Apr\_2014\_08\_57\_00.mp3
\item Wacuri Business call inc. Journey by Brooks.mp3
\item Wacuri Business Journey inc. disc. of Mindfulness for Wacuri.mp3
\item ---To-Be-Edited-for-MVP
\item \_List-of-Journeys-091514.jpg
\item Journey to a Bumblebee (2).mp3
\item Journey to a Comet.mp3
\item Journey to a Fallen Leaf.mp3
\item Journey to A Flame.mp3
\item Journey to a Future Self.mp3
\item Journey to a Galaxy Cluster.mp3
\item Journey to a New Leaf.mp3
\item Journey to a Photon(1).mp3
\item Journey to a Rainbow.mp3
\item Journey to a Rose.mp3
\item Journey to a Spider Web.mp3
\item Journey to a Wisp of Smoke.mp3
\item Journey to an Asteriod.mp3
\item Journey to an Atom.mp3
\item Journey to Ancient Bacteria .mp3
\item Journey to Bioluminessence.mp3
\item Journey to Christ Energy -.mp3
\item Journey to Dark Matter and Dark Energy.mp3
\item Journey to Entrainment, Resonance and Transmission.mp3
\item Journey to Forgiveness.mp3
\item Journey to Money as Light.mp3
\item Journey to Nature Spirits.mp3
\item journey to nature.mp3
\item Journey to our Feet.mp3
\item Journey to Photosynthesis.mp3
\item Journey to Resonance.mp3
\item Journey to Resonant Harmony.mp3
\item Journey to Silence and Breath.mp3
\item Journey to Space.mp3
\item Journey to the Amazon Basin.mp3
\item Journey to the Aura of all Things .mp3
\item Journey to the Beating of Wings.mp3
\item Journey to the Birth of Stars.mp3
\item Journey to the Blessings of All Things.mp3
\item Journey to the Body.mp3
\item Journey to the Canopy of Stars.mp3
\item Journey to the Center of Your Own Being.mp3
\item Journey to the Community of Jellyfish.mp3
\item Journey to the Consciousness of all Things.mp3
\item Journey to the Consciousness of Cells.mp3
\item Journey to the Cosmic Gainer.mp3
\item Journey to The Creature (Vibrational Tone).mp3
\item Journey to The Creature.mp3
\item Journey to the Digital Experience.mp3
\item Journey to the Edge of Sleep.mp3
\item Journey to the Emergence of a Flower.mp3
\item Journey to the Essence of a Seed.mp3
\item Journey To The Essence Of Oxygen.mp3
\item Journey to the Eternal Now.mp3
\item Journey to the Exuberance of Youth.mp3
\item Journey to the Fractal Universe.mp3
\item Journey to the Grace of the Normal Human Body (B led).mp3
\item Journey to the Idea of Death.mp3
\item Journey to the Illusion of Time .mp3
\item Journey to the Kingdom of Mosses and Lichen.mp3
\item Journey to the Limina.mp3
\item Journey to the Magentic Field of the Earth.mp3
\item Journey to the Mineral Kingdom.mp3
\item Journey to the Nested Hierarchy of All Things.mp3
\item Journey to the Newly Born.mp3
\item Journey to the Northern Lights.mp3
\item Journey to the Other.mp3
\item Journey to the Plant Kingdom (2).mp3
\item Journey to the Quantum Brain.mp3
\item Journey to the Reflection of Moonlight.mp3
\item Journey to the Rhythmns of Time.mp3
\item Journey to the Rythmn of the Seasons.mp3
\item Journey to the Shifting of Creatures.mp3
\item Journey to the Silence of the Forest.mov
\item Journey to the Song of Owls.mp3
\item Journey to the Soul of Holo (private).mp3
\item Journey to the Soul of Ingrid (private).mp3
\item Journey to the Sound of Music Everywhere.mp3
\item Journey to the Sound of Rain.mp3
\item Journey to the Spirit of A Mother.mp3
\item Journey to the Spirit of Beauty in all things .mp3
\item Journey to the Spirit of Beauty.mp3
\item Journey to the Spirit of Buddha.mp3
\item Journey to the Spirit of Celebration.mp3
\item Journey to The Spirit of Childhood.mp3
\item Journey to the Spirit of Creativity.mp3
\item Journey to the Spirit of Family.mp3
\item Journey to the Spirit of Peace.mp3
\item Journey to the Spirit of Robin Williams.mp3
\item Journey to the Spirit of Transformation (Death).mp3
\item Journey to the Surface of Mars.mp3
\item Journey to the Thousand Petalled Lotus Flower.mp3
\item Journey to the Tip of Your Tongue-Memories.mp3
\item Journey to the Transformation from Hectic to Harmony.mp3
\item Journey to the Transformation of Confusion to Clarity.mp3
\item Journey to the Transformation of Fear.mp3
\item Journey to the Transformation of Sadness and Ennui.mp3
\item Journey to the Transformation of Violence.mp3
\item Journey to the Transformation of Worry.mp3
\item Journey to the Universal Mind.mp3
\item Journey to the Universal Womb.mp3
\item Journey to the Wacuri Creature.mp3
\item Journey to Water Underground.mp3
\item Journey to Your Biochemical Cascade.mp3
\item Journey to Your Mitochondria.mp3
\item Journey to Your Monkey Mind .mp3
\item Journey to Your Neural Network.mp3
\item Journey to Your Skeletal Structure.mp3
\item Journey to Your Soul Group.mp3
\item Journey to Yourself in the Womb.mp3
\item Journey-to-the-Elders.mp3
\item Shareable
\item A Galaxy Cluster.mp3
\item Forgiveness (1).mp3
\item Journey to a Bumblebee (2).mp3
\item Journey to Ancient Bacteria .mp3
\item Journey to Bioluminessence.mp3
\item Journey to the Birth of Stars.mp3
\item Journey to the Illusion of Time .mp3
\item Journey to the Transformation of Fear.mp3
\item Journey-to-the-Elders.mp3
\end{enumerate}

\end{comment}
  \end{document}


%% look up Swift Trust Theory
%% Add a good Journey Journal table.
%% TODO: Write Curiosity Table
%% TODO: Edit Henry's work in the origin chapter.


